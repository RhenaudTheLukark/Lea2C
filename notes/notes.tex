\documentclass[12pt, a4paper]{article}

\usepackage[utf8]{inputenc}
\usepackage[T1]{fontenc}
\usepackage[francais]{babel}
\usepackage[most]{tcolorbox}
\usepackage{helvet}

\renewcommand{\familydefault}{\sfdefault}




\definecolor{block-gray}{gray}{0.65}
\newtcolorbox{codequote}{colback=block-gray,grow to right by=-10mm,grow to left by=-10mm,
boxrule=0pt,boxsep=0pt,breakable}



\begin{document}
\hbadness=10000


\begin{center}
    \rule{\linewidth}{0.5mm} \\
    \Large\textbf{Projet Compilation 2020\\}
    Rhenaud Dubois\\
    Loïc Lachiver\\
    Nicolas Marin-Pache\\
    Romain Pigret-Cadou
    
    \rule{\linewidth}{0.5mm} \\
\end{center}



%%%%%%%%%%%%%%%%%%%%%%%%%%%%%%%%%%%%%%%%%%%%%%%%%%%%%%%
%%%%%%%%%%%%%%%%%%%%%%%%%%%%%%%%%%%%%%%%%%%%%%%%%%%%%%%
%%%%%%%%%%%%%%%%%%%%%%%%%%%%%%%%%%%%%%%%%%%%%%%%%%%%%%%
\subsection*{Objectifs}
L'objectif de ce projet est de presque finaliser la création d'un langage de programmation. L'essentiel du processus étant déjà fait, nous n'avions qu'à implémenter la génération d'un code en C fonctionnel pour les structures "if else", "for", "while" et "switch".
\\
L'idée est de générer un code C mais en utilisant uniquement des "goto" et des "if" sur une valeur et sans "else". En effet il sera ensuite plus facile de traduire ce code en assembleur.


%%%%%%%%%%%%%%%%%%%%%%%%%%%%%%%%%%%%%%%%%%%%%%%%%%%%%%%
%%%%%%%%%%%%%%%%%%%%%%%%%%%%%%%%%%%%%%%%%%%%%%%%%%%%%%%
%%%%%%%%%%%%%%%%%%%%%%%%%%%%%%%%%%%%%%%%%%%%%%%%%%%%%%%
\subsection*{Implémentation et difficultés}
\textbf{Implémentation du IF ELSE:}

\begin{codequote}
    \begin{verbatim}
        if(XXX) {
            YYY
        } else {
            ZZZ
        }
    \end{verbatim}
\end{codequote}
Code à générer :
\begin{codequote}
    \begin{verbatim}
        int _t_ = XXX;
        if (_t_) goto label_then;
        label_else:
        ZZZ
        goto label_endif;
        label_then:
        YYY
        label_endif:
    \end{verbatim}
\end{codequote}
On notera que label\_else est seulement utile à la compréhension.\\

\textbf{Implémentation du WHILE:}
\begin{codequote}
    \begin{verbatim}
        while (XXX) {
            YYY
        }
    \end{verbatim}
\end{codequote}
Code à générer :
\begin{codequote}
    \begin{verbatim}
        label_begin:
        int _t_ = XXX;
        if(_t_) goto label_then;
        goto label_end;
        label_then:
        YYY
        goto label_begin;
        label_end:
    \end{verbatim}
\end{codequote}


\textbf{Implémentation du FOR:}
\begin{codequote}
    \begin{verbatim}
        for( XXX ; YYY ; ZZZ) {
            AAA
        }
    \end{verbatim}
\end{codequote}
Code à générer :
\begin{codequote}
    \begin{verbatim}
        XXX
        label_test:
        int _t_ = YYY;
        if(_t_) goto label_then;
        goto label_end;
        label_then:
        AAA
        ZZZ
        goto label_test;
        label_end:
    \end{verbatim}
\end{codequote}
Le statement d'initialisation est effectué au début et dans tous les cas. Le test correspond à 'label_test' et est exécuté à chaque fin de boucle. Si le test est valide, on exécute AAA ZZZ, sinon on se rend à label\_end et le for est terminé. Chaque 'for' possède un id généré par le Java qui est ajouté à chaque label, on peut ainsi imbriquer autant de for que l'on souhaite sans craindre de conflits d'étiquettes.\\

\textbf{Implémentation du SWITCH:}
\begin{codequote}
    \begin{verbatim}
        switch (XXX){
            case YYY_A: ZZZ_A
            case YYY_B: ZZZ_B
            case YYY_C: ZZZ_C
            ...
            default: ZZZ_0 
        }
    \end{verbatim}
\end{codequote}
Code à générer :
\begin{codequote}
    \begin{verbatim}
        int _t_;
        _t_ = (XXX != YYY_A)
        if(_t_) goto label_B;
        ZZZ_A
        goto label_end;
        label_B:
        _t_ = (XXX != YYY_B)
        if(_t_) goto label_C;
        ZZZ_B
        goto label_end;
        label_C:
        _t_ = (XXX != YYY_C)
        if(_t_) goto label_D;
        ZZZ_C
        goto label_end;
        label_D:
        ...
        ZZZ_0
        label_end:
    \end{verbatim}
\end{codequote}
On remarquera que notre switch ne peut pas grouper des cas, car il y a un "break" automatique.

%%%%%%%%%%%%%%%%%%%%%%%%%%%%%%%%%%%%%%%%%%%%%%%%%%%%%%%
%%%%%%%%%%%%%%%%%%%%%%%%%%%%%%%%%%%%%%%%%%%%%%%%%%%%%%%
%%%%%%%%%%%%%%%%%%%%%%%%%%%%%%%%%%%%%%%%%%%%%%%%%%%%%%%
\subsection*{Remarques}


\end{document}
